%%%%%%%%%%%%%%%%%%%%%%%%%%%%%%%%%%%%%%%%%%%%%%%%%%%%%%%%%%%%
%%% LIVECOMS ARTICLE TEMPLATE FOR BEST PRACTICES GUIDE
%%% ADAPTED FROM ELIFE ARTICLE TEMPLATE (8/10/2017)
%%%%%%%%%%%%%%%%%%%%%%%%%%%%%%%%%%%%%%%%%%%%%%%%%%%%%%%%%%%%
%%% PREAMBLE
\documentclass[9pt,bestpractices]{livecoms}
% Use the 'onehalfspacing' option for 1.5 line spacing
% Use the 'doublespacing' option for 2.0 line spacing
% Use the 'lineno' option for adding line numbers.
% The 'bestpractices' option for indicates that this is a best practices guide.
% Omit the bestpractices option to remove the marking as a LiveCoMS paper.
% Please note that these options may affect formatting.

\usepackage{lipsum} % Required to insert dummy text
\usepackage[version=4]{mhchem}
\usepackage{siunitx}
\DeclareSIUnit\Molar{M}
\usepackage[italic]{mathastext}
\graphicspath{{figures/}}

%%%%%%%%%%%%%%%%%%%%%%%%%%%%%%%%%%%%%%%%%%%%%%%%%%%%%%%%%%%%%%%%%%%%%
%% Place any additional macros here.  Please use \newcommand* where
%% possible
%%%%%%%%%%%%%%%%%%%%%%%%%%%%%%%%%%%%%%%%%%%%%%%%%%%%%%%%%%%%%%%%%%%%%
\usepackage[colorinlistoftodos]{todonotes}

%%%%%%%%%%%%%%%%%%%%%%%%%%%%%%%%%%%%%%%%%%%%%%%%%%%%%%%%%%%%
%%% IMPORTANT USER CONFIGURATION
%%%%%%%%%%%%%%%%%%%%%%%%%%%%%%%%%%%%%%%%%%%%%%%%%%%%%%%%%%%%

\newcommand{\versionnumber}{0.1}  % you should update the minor version number in preprints and major version number of submissions.
\newcommand{\githubrepository}{\url{https://github.com/michellab/BioMolSetupPaper}}  %this should be the main github repository for this article

%%%%%%%%%%%%%%%%%%%%%%%%%%%%%%%%%%%%%%%%%%%%%%%%%%%%%%%%%%%%
%%% ARTICLE SETUP
%%%%%%%%%%%%%%%%%%%%%%%%%%%%%%%%%%%%%%%%%%%%%%%%%%%%%%%%%%%%
\title{Best Practices for Biomolecular Simulation Setup : v\versionnumber}

\author[1*]{Rommie E. Amaro}
\author[2*]{John D. Chodera}
\author[3*]{David L. Mobley}
\author[4*]{Antonia S. J. S. Mey}
\author[4*]{Julien Michel}
\affil[1]{Institution 1}
\affil[2]{Computational and Systems Biology Program, Memorial Sloan Kettering Cancer Center, New York NY 10065}
\affil[3]{Departments of Pharmaceutical Sciences and Chemistry, University of California, Irvine}
\affil[4]{EaStCHEM School of Chemistry, David Brewster road, Joseph Black Building, The King's Buildings, Edinburgh, EH9 3FJ, UK}

\corr{email1@example.com}{REA}  % Correspondence emails.  FMS and FS are the appropriate authors initials.
\corr{john.chodera@choderalab.org}{JDC}
\corr{dmobley@mobleylab.org}{DLM}
\corr{antonia.mey@ed.ac.uk}{ASJSM}
\corr{info@julienmichel.net}{JM}

\contrib[\authfn{1}]{These authors contributed equally to this work}
\contrib[\authfn{2}]{These authors also contributed equally to this work}

\presentadd[\authfn{3}]{Department, Institute, Country}
\presentadd[\authfn{4}]{Department, Institute, Country}

\blurb{This LiveCoMS document is maintained online on GitHub at \githubrepository; to provide feedback, suggestions, or help improve it, please visit the GitHub repository and participate via the issue tracker.}

%%%%%%%%%%%%%%%%%%%%%%%%%%%%%%%%%%%%%%%%%%%%%%%%%%%%%%%%%%%%
%%% ARTICLE START
%%%%%%%%%%%%%%%%%%%%%%%%%%%%%%%%%%%%%%%%%%%%%%%%%%%%%%%%%%%%

\begin{document}

\begin{frontmatter}
\maketitle

\begin{abstract}
Simulations of biomolecular systems are complex and often demanding but can provide great insights and valuable quantitative predictions. 
However, these simulations require considerable human expertise to set up well, and key aspects of system preparation and setup are often overlooked for a variety of reasons.
Thus, simulations of the ``same'' biomolecular system prepared by different individuals or different software tools may yield substantially different results due to different choices made in system preparation, such as different handling of crystallographic waters, different treatment of missing residues or loops in a protein, or different choices for a ligand or receptor's protonation state. 
In some cases, human error can also play a role, with a user perhaps forgetting to check for disulfide bonds. 
Here, we identify the issues we believe are most important for users and/or software tools to consider when preparing biomolecular systems for simulation and present a checklist for users to follow to ensure they have handled the most crucial or troublesome aspects of system preparation, with recommendations for how to handle each step in the process when best practices have been established by the literature.
We also present a detailed explanation of the items in our checklist, including why we make the recommendations we do.
Additionally, because a useful checklist must necessarily be brief, we provide a discussion of other important factors to consider which did not make our major checklist. 
We believe this work provides a starting point to help reduce the number of simulations run which are flawed due to setup failures, and we hope the community will contribute to updating and improving this document to that end.
\end{abstract}

\end{frontmatter}
% List TODOs here
\todototoc
\listoftodos

%%%%%%%%%%%%%%%%%%%%
%              Introduction                  %
%%%%%%%%%%%%%%%%%%%%
\section{Introduction and Goals}
\todo[inline, color={yellow!20}]{ASJSM: @Volunteer write actual intro}
This work focuses on providing a brief checklist of the most critical and most frequently overlooked items in the preparation of biomolecular simulations, so that that practitioners can follow this checklist, and reviewers can potentially use it in evaluating manuscripts and assessing whether method descriptions are complete. 
We also focus on highlighting issues where further clarity may be needed, either via additional study, or from work of which we are unaware.
Additionally, we value input from the community on how best to handle the issues raised, and others we may have overlooked, as an additional goal of this work is to develop a set of \emph{community} best practices.
In addition to the provided checklist, we provide an explanation of why the checklist includes the items it does and why we make the recommendations we do, in order to further facilitate training and community input.

This work can perhaps also be useful for authors who are writing papers in order to help determine what details to report in their Methods section -- specifically, such sections should at minimum address the details highlighted in this document. 
%How to write a methods section by John Chodera: http://klogw.org/2016/07/28/how-to-write-a-methods-section/

Our goal in this work is to \emph{inform} about what we believe are best practices for preparation in this area and provide an aid to authors and reviewers; we have no intention of providing a set of standards which should be \emph{enforced}. 
Indeed, there may be good reasons from deviating from recommended practices in certain cases; in our view, authors should simply note any deviations and explain why they made the choices they did. 


%%%%%%%%%%%%%%%%%%%%
%                  Scope                      %
%%%%%%%%%%%%%%%%%%%%
\section{Prerequisites and Scope}
\todo[inline, color={yellow!20}]{ASJSM: @Volunteer write actual prerequesites/scopes}
Our focus in this work is on what is often the earliest stage of preparation of biomolecular simulations -- deciding what system to simulate and which molecules and atoms are in it in which positions. 
We deliberately avoid the issue of assigning (and potentially developing) force field parameters, as this involves many additional complexities that warrant their own ``best practices'' document.
We also, at least for now, focus on protein or protein-ligand systems, avoiding nucleic acids such as DNA or RNA as we lack the requisite expertise (though this work may be extended later to cover such cases). 
Additionally, we avoid membrane proteins for similar reasons.
Thus our overall focus is on preparation for \emph{relatively simple biomolecular simulations of soluble proteins that may include non-covalent small molecule ligands}, and especially on the force field agnostic aspects of preparation.
We expect to eventually include handling of cofactors, though this is not within our initial scope.

Thus, our scope \emph{does not include} assignment of parameters from force field libraries, and especially generation of parameters (such as for unusual nonstandard residues where literature parameters may not be available, or for covalent cofactors, complex ligands or factors, etc.). 
This is both because these may be research questions, and because the particular approaches applied will often be coupled to the choice of force field to a significant extent.
Additionally, we do not include any aspects of choice of simulation protocol such as minimization, cutoffs, and other factors as, in our view, these all come significantly downstream of system preparation. 


%%%%%%%%%%%%%%%%%%%%
%                  Checklist                  %
%%%%%%%%%%%%%%%%%%%%
\section{Checklist}
\todo[inline, color={yellow!20}]{ASJSM: @Volunteer write intro to checklists}
\todo[inline, color={yellow!20}]{ASJSM: Tidy up checklist}

\begin{Checklists*}[p!]

\begin{checklist}{Step 0 -- Know what you want to simulate }
\textbf{What are the first questions that need addressing before setting up a molecular dynamics simulation}\\
Extensive explanation for the checklist questions can be found in section~\ref{sec:stage0}.
\begin{itemize}
\item What is the goal of your simulation(s)? 
\item Can you hope to achieve those goals with this simulation?
\item What do you know (or expect) about the potential timescales in your system, and can you hope to capture those with this simulation?
\item Will you be able to gather good enough statistics from this study, based on what you know about those timescales?
\item What will be the costs of your simulation (computer time, storage, memory, wallclock, network bandwidth, data throughput) and can you afford those costs? 
\item What are the experimental conditions (including salt conditions, buffers, cofactors, etc.)? 
\item Do you want to match those conditions in your simulation? If not, why not?
\end{itemize}
\end{checklist}

\begin{checklist}{Stage 1 -- Protein Setup}
\textbf{What are the key issues to be aware of when setting up a protein simulation?}\\
Extensive explanation for the checklist questions can be found in section~\ref{sec:stage1}.
\begin{enumerate}
\item Have you identified the right protein sequence?
\item What are your simulation conditions (Assays, crystal structure, intended) \todo[inline, color={yellow!20}]{ASJSM: @Chodera/Amaro simulation conditions}
\item Structure selection (Mey) 
    \begin{itemize}
    \item  X-ray structural data
    \item Dealing with NMR structural ensembles
    \item Other sources of structural data: e.g. cryo-EM
    \item Monomoer or oligomer?
    \end{itemize}
\item Have you checked for disulfide bonds depending on reducing/oxidizing environments? (Amaro)
\item Have you checked for post-translational modifications? (Chodera)
\item Have you modeled in missing residues and loops? (Chodera/Amaro tag-team)
\item Have you assign protein protonation states/tautomers via a consistent, well-documented, and (ideally) reproducible procedure? 
\item Have you considered crystal waters ?(Amaro, Mey)
\item Are there any metal ions in the system that need attention? (JDC)
\end{enumerate}
\end{checklist}
\end{Checklists*}

\begin{Checklists*}[p!]


\begin{checklist}{Stage 2 -- Ligand Setup}
\textbf{What are the key issues to be aware of when setting up a small molecule simulation, in complex with a protein or on its own?}\\
Extensive explanation for the checklist questions can be found in section~\ref{sec:stage2}.
\begin{itemize}
\item Do I have a ligand or other small molecule in my simulation?
\end{itemize}
\textbf{If yes}:
\begin{itemize}
\item Select the correct ligand binding mode %(Greg Warren, Bob Tolbert, Mobley) % Electron density, ligand binding mode, etc. 
    \begin{itemize}
    \item Are you sure you have the right ligand?
    \item If the starting point is crystallographic, does the electron density support its presence? 
    \item Is it present only because of crystal contacts or packing issues?
    \item If you are modeling the ligand in, how will you place it? %(Get help from Amaro)
Will you use docking, shape overlays, knowledge of binding modes of similar ligands, or other factors?
    \item If you have modeled it in, how will you determine whether this is the correct binding mode? Might other binding modes be possible? How would this affect your results?
    \item Will you need to run simulations of the ligands to validate the putative binding mode?
    \end{itemize}
   \item Do you have the correct tautomer? 
   \item Do you have the right protonation state?
\end{itemize}
\end{checklist}

\begin{checklist}{Stage 3 -- Solvent and Counterions}
\textbf{Things to be aware of when dealing with solvent and counter ions.}\\
Extensive explanation for the checklist questions can be found in section~\ref{sec:stage3}.
\begin{itemize}
\item Do you use a different solvent than water?
\item What is the ionic strength at which you want to simulate and what counterions are required?  (Mobley)
\end{itemize}
\end{checklist}

\end{Checklists*}



%%%%%%%%%%%%%%%%%%%%
%            Checklist in Detail            %
%%%%%%%%%%%%%%%%%%%%
\section{Detailed explanation of checklist items}
\subsection{Know what you want to simulate}
\label{sec:stage0}

%%%%%%%%%%%%%%%%%%%%
%           Protein setup                   %
%%%%%%%%%%%%%%%%%%%%
\subsection{Protein Setup}
\label{sec:stage1}
% On protonation states:
%Gianni de Fabritiis has done some nice work on this recently -- 10.1021/acs.jctc.7b00480
%Oliver Beckstein has looked fairly extensively at using proPKA for protein prep -- see https://github.com/Becksteinlab/propkatraj
%Marilyn Gunner has extensively explored the assignment of protonation states on both proteins and now ligands using MCCE2
\subsubsection{Structural selection}
\todo[inline, color={yellow!20}]{ASJSM: @ASJM write}
Termini: Build them, or cap? Depends on length; if too long can contribute to timescales. Write some guidelines into how to know. %(Mey)
\subsubsection{Disulphide bridges}
\todo[inline, color={yellow!20}]{ASJSM: @REA write}
\subsubsection{posttranslational modifications}
\todo[inline, color={yellow!20}]{ASJSM: @JDC write}
\subsubsection{Missing residues}
\todo[inline, color={yellow!20}]{ASJSM: @JDC and @REA write}
    \begin{itemize}
    \item Schrodinger tools model in short loops, but truncate long loops with proton caps (?)
    \item UCSF Modeller terrible for loops without care; Rosetta Model works well ? (JDC says)
    \item (But Amaro reports success stories with Schrodinger tools if one knows when to use them and does not try to push the tool beyond the limit of reasonable application - eg loops < 10-12 residues, otherwise need templated homology models)
    \end{itemize}
    
 \subsubsection{Protonation states}
 \todo[inline, color={yellow!20}]{ASJSM: @JDC and @DLM write}
     \begin{itemize}
    \item PROPKA may be a good automated tool [ref] though it can be sensitive to the specific snapshot chosen for analysis, so propkatraj (\url{https://github.com/Becksteinlab/propkatraj}) may be worth using to provide a more thorough analysis
    \item Multi-Conformer Continuum Electrostatics (MCCE) is another well-established option [ref] 
    \item Constant pH MD simulations may offer an alternative as they develop further [ref  Case, McCammon, Roitberg, Shen, Roux, Chodera]
    \item Careful hand selection may be needed in some cases but must be carefully documented and is too onerous to be recommended as the only solution (that is, we recommend hand selection be applied only after applying an algorithmic approach)
    \end{itemize}
    
\subsubsection{Crystal waters}
\todo[inline, color={yellow!20}]{ASJSM: @ASJM and @REA write}
 
\subsubsection{Metal ions}
\todo[inline, color={yellow!20}]{ASJSM: @JDC write}

%%%%%%%%%%%%%%%%%%%%
%            Ligand setup                   %
%%%%%%%%%%%%%%%%%%%%
\subsection{Ligand Setup}
\label{sec:stage2}

%Electron density, ligand binding mode, etc. (Greg Warren, Bob Tolbert, Mobley)
%Is this the right ligand?
%Does the density support?
%Is that only because of crystal contacts?
%How do you put your new ligand in? (Get help from Amaro)
%...docking, shape overlay, etc
%Run dynamics of ligands?
\subsubsection{Electron densities}
\todo[inline, color={yellow!20}]{ASJSM: @DLM write}

\subsubsection{Docking/Binding modes}
\todo[inline, color={yellow!20}]{ASJSM: @REA and @ASJSM and @DLM @JM write}

\subsubsection{Tuatomer}
\todo[inline, color={yellow!20}]{ASJSM: @JDC write}

\subsubsection{Protonation}
\todo[inline, color={yellow!20}]{ASJSM: @JDC write}
While simple proteins without covalent modifications or nonstandard residues can typically be handled by library force fields for biomolecular simulations, ligands will involve parameter assignment.
As noted above, parameterization is not within the scope of this work, but it is important to note that there are roughly two categories of approaches for handling of ligand parameter assignment: 
General-purpose small molecule force fields (e.g. GAFF, GAFF2, OPLS2, OPLS3, etc.) %does CHARMM belong here?
which provide generic parameters that can easily be assigned to the vast majority of small organic molecules, possibly with a partial charge calculation required; and more advanced force fields which require parameter development for each new molecule to be considered (e.g. AMOEBA, ... ??). 

Handling of covalently bound cofactors, adducts, and certain nonstandard residues (those without library parameters available) can be an even more complex issue than handling of ligand parameterization, often turning into a research topic requiring considerable care. 

%%%%%%%%%%%%%%%%%%%%
%      Solvent/Counterions              %
%%%%%%%%%%%%%%%%%%%%
\subsection{Solvent and Counterions}%
\label{sec:stage3}%
\todo[inline, color={yellow!20}]{ASJSM: @REA and @DLM write}%
Ultimately, finalizing certain aspects of setup will require selection of water and ion models, which is often coupled (at least partially) to the choice of force field.%
The choice of these models is outside our scope, but is important to note. %
Many different water models are available, with varying degrees of quality for pure solvent properties.%
The choice of water model is also coupled to the choice of model for ions -- for example, different water models can require different models for monatomic ions in order to prevent aggregation of ions [refs].%
%
%%%%%%%%%%%%%%%%%%%%
%      Non checklist stuff                %
%%%%%%%%%%%%%%%%%%%%
\section{Other things to think about that didn't make the checklist}
(minor) Select simulation box size (Mey) -- important, but first thing to check in analysis.%
Here, we focus on items which are also important part of system preparation, but which are less frequent causes of critical failures or are not as often or easily overlooked.
(minor) Add solvent then ions, or ions then solvent? Pre-equilibrate ion-water mixture? (Amaro)%
Determine electrostatic potential around molecule and place ions at minima (default AMBER approach does this in solvent, replacing some solvent with ions)%
%
%%%%%%%%%%%%%%%%%%%%
%           Conclusions                     %
%%%%%%%%%%%%%%%%%%%%
\section{Conclusions}
%
\section*{Key terminology}
This section discusses all key terms used above.%
%
%%%%%%%%%%%%%%%%%%%%
%       Acknowledgements              %
%%%%%%%%%%%%%%%%%%%%
\section*{Acknowledgements}
%
\section*{Author Contributions}
%%%%%%%%%%%%%%%%
% This section mustt describe the actual contributions of
% author. Since this is an electronic-only journal, there is
% no length limit when you describe the authors' contributions,
% so we recommend describing what they actually did rather than
% simply categorizing them in a small number of
% predefined roles as might be done in other journals.
%
% See the policies ``Policies on Authorship'' section of https://livecoms.github.io
% for more information on deciding on authorship and author order.
%%%%%%%%%%%%%%%%

(Explain the contributions of the different authors here)

% We suggest you preserve this comment:
For a more detailed description of author contributions,
see the GitHub issue tracking and changelog at \githubrepository.

\section*{Other Contributions}
%%%%%%%%%%%%%%%
% You should include all people who have filed issues that were
% accepted into the paper, or that upon discussion altered what was in the paper.
% Multiple significant contributions might mean that the contributor
% should be moved to authorship at the discretion of the a
%
% See the policies ``Policies on Authorship'' section of https://livecoms.github.io for
% more information on deciding on authorship and author order.
%%%%%%%%%%%%%%%

(Explain the contributions of any non-author contributors here)
% We suggest you preserve this comment:
For a more detailed description of contributions from the community and others, see the GitHub issue tracking and changelog at \githubrepository.

\section*{Potentially Conflicting Interests}
%%%%%%%
%Declare any potentially competing interests, financial or otherwise
%%%%%%%

Declare any potentially conflicting interests here, whether or not they pose an actual conflict in your view.

\section*{Funding Information}
%%%%%%%
% Authors should acknowledge funding sources here. Reference specific grants.
%%%%%%%
FMS acknowledges the support of NSF grant CHE-1111111.

\bibliography{livecoms-sample}

%%%%%%%%%%%%%%%%%%%%%%%%%%%%%%%%%%%%%%%%%%%%%%%%%%%%%%%%%%%%
%%% APPENDICES
%%%%%%%%%%%%%%%%%%%%%%%%%%%%%%%%%%%%%%%%%%%%%%%%%%%%%%%%%%%%

%\appendix


\end{document}
